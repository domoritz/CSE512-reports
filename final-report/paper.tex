\documentclass{chi2009}
\usepackage{times}
\usepackage{url}
\usepackage{graphics}
\usepackage{color}
\usepackage[pdftex]{hyperref}
\hypersetup{%
pdftitle={Visualizing Physical Query Execution in a Relational Big Data Management System},
pdfauthor={Umar Javed, Thierry Moreau, Dominik Moritz, Adriana Szekeres},
pdfkeywords={},
bookmarksnumbered,
pdfstartview={FitH},
colorlinks,
citecolor=black,
filecolor=black,
linkcolor=black,
urlcolor=black,
breaklinks=true,
}
\newcommand{\comment}[1]{}
\definecolor{Orange}{rgb}{1,0.5,0}
\newcommand{\todo}[1]{\textsf{\textbf{\textcolor{Orange}{[[#1]]}}}}

\pagenumbering{arabic}  % Arabic page numbers for submission.  Remove this line to eliminate page numbers for the camera ready copy

\begin{document}
% to make various LaTeX processors do the right thing with page size
\special{papersize=8.5in,11in}
\setlength{\paperheight}{11in}
\setlength{\paperwidth}{8.5in}
\setlength{\pdfpageheight}{\paperheight}
\setlength{\pdfpagewidth}{\paperwidth}

% use this command to override the default ACM copyright statement
% (e.g. for preprints). Remove for camera ready copy.
%\toappear{Submitted for review to CHI 2009.}
\toappear{}

\title{Visualizing Physical Query Execution in a Relational Big Data Management System}
\numberofauthors{1}
\author{\alignauthor Umar Javed, Thierry Moreau, Dominik Moritz, Adriana Szekeres \\
\affaddr{Dept. of Computer Science, University of Washington} \\ \affaddr{ Seattle, Washington, USA} \\
\email{\texttt{\{ujaved, moreau, domoritz, aaasz\}@cs.washington.edu}}
}

\maketitle

\begin{abstract}
We propose a visualization system for understanding and exploring query execution and data movement in a distributed database system. Our tool will provide insight into: (1) data flow between query fragments and between workers, (2) query execution and operator dependencies, (3) cluster utilization, (4) network utilization. Our tool will help developers understand and improve query execution, and will provide insight into common problems such as data skew or performance bottlenecks.

In particular, our visualization system is built to inspect query execution in Myria\footnote{\url{http://myria.cs.washington.edu/}}, a distributed big data management system currently being developed in the UW CSE database group. Myria aims towards building a distributed database platform to provide \emph{big data management and analytics as a service} primarily for scientific applications.

Our vision is to build a generic visualization that can be used for other systems as well (e.g. Spark, Hadoop). This project builds on an existing prototype that uses logs generated during query execution to create a time series visualization of operator states.

\end{abstract}

%\keywords{put author keywords here}

%\category{H.5.2}{Information Interfaces and Presentation}{Miscellaneous}[Optional sub-category]

\section{Introduction}

\begin{itemize}
    \item problem: hard to learn why a query is slow
    \item motivation: help developers/programmers dig what the problems might be
\end{itemize}

% Umar

\section{Related Work}

% Dom


\section{Approach}

% Adriana

% What the user sees
% Why did we choose the visualization
% How it is implemented

\begin{itemize}
    \item determine possible causes (data skew, stragglers, etc.) for why a query might be slow and design visualizations to
help
    \item we designed three views - physical query plan, fragment view, communication view.
\end{itemize}

\subsection{Back-end}

% Dom

\subsection{Front-end}

% Thierry

\subsubsection{Physical query plan view}

% Thierry

\subsubsection{Fragment execution view}

% Adriana

\subsubsection{Worker communication view}

% Umar

\section{Evaluation}

% How it helps
% Examples

% Thierry + Dom

\textbf{Discussion:}

\section{Conclusion}

% Adriana

\section{Future Work}

% Dom

\bibliographystyle{abbrv}
\bibliography{sample}

\end{document}
