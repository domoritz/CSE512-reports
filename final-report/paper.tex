\documentclass{chi2009}
\usepackage{times}
\usepackage{url}
\usepackage{graphics}
\usepackage{color}
\usepackage[pdftex]{hyperref}
\usepackage{xspace}

\newcommand*{\system}{OGEN\xspace}
\newcommand*{\papertitle}{\system: Visualizing Physical Query Execution in a Relational Big Data Management System}
\newcommand*{\graph}{Physical Query Plan\xspace}
\newcommand*{\fragment}{Fragment Execution\xspace}
\newcommand*{\network}{Worker Communication\xspace}
\newcommand*{\overall}{Fragment Overall\xspace}

\hypersetup{%
pdftitle={\papertitle},
pdfauthor={Umar Javed, Thierry Moreau, Dominik Moritz, Adriana Szekeres},
pdfkeywords={},
bookmarksnumbered,
pdfstartview={FitH},
colorlinks,
citecolor=black,
filecolor=black,
linkcolor=black,
urlcolor=black,
breaklinks=true,
}
\newcommand{\comment}[1]{}
\definecolor{Orange}{rgb}{1,0.5,0}
\newcommand{\todo}[1]{\textsf{\textbf{\textcolor{Orange}{[[#1]]}}}}

\pagenumbering{arabic}  % Arabic page numbers for submission.  Remove this line to eliminate page numbers for the camera ready copy

\begin{document}
% to make various LaTeX processors do the right thing with page size
\special{papersize=8.5in,11in}
\setlength{\paperheight}{11in}
\setlength{\paperwidth}{8.5in}
\setlength{\pdfpageheight}{\paperheight}
\setlength{\pdfpagewidth}{\paperwidth}

% use this command to override the default ACM copyright statement
% (e.g. for preprints). Remove for camera ready copy.
%\toappear{Submitted for review to CHI 2009.}
\toappear{}

\title{\papertitle}
\numberofauthors{1}
\author{\alignauthor Umar Javed, Thierry Moreau, Dominik Moritz, Adriana Szekeres \\
\affaddr{Dept. of Computer Science, University of Washington} \\ \affaddr{ Seattle, Washington, USA} \\
\email{\texttt{\{ujaved, moreau, domoritz, aaasz\}@cs.washington.edu}}
}

\maketitle

\begin{abstract}

We propose a visualization system for understanding and exploring query execution and data movement in a distributed database management system (DDBMS). Our tool will provide insight into: (1) data flow between query fragments and between workers, (2) query execution and operator dependencies, (3) cluster utilization, (4) network utilization. Our tool, which we call \system, will help developers understand and improve query execution, and will provide insight into common problems such as data skew or performance bottlenecks.

In particular, \system is built to inspect query execution in Myria\footnote{\url{http://myria.cs.washington.edu/}}, a distributed big data management system currently being developed in the UW CSE database group. Myria aims towards building a distributed database platform to provide \emph{big data management and analytics as a service} primarily for scientific applications.

The proposed visualizations can easily be applied to other DDBMS as well (e.g. Spark, Hadoop).

\end{abstract}

%\keywords{put author keywords here}

%\category{H.5.2}{Information Interfaces and Presentation}{Miscellaneous}[Optional sub-category]

\section{Introduction}

% Umar

% What is myria, how does it work (look at report from last quarter but shorter), which languages are supported
% motivation
% questions
% why is a visualization a good way to answer the questions

\begin{itemize}
    \item problem: hard to learn why a query is slow
    \item motivation: help developers/programmers dig what the problems might be
\end{itemize}


\section{Related Work}

Twitter developed Ambrose\cite{ambrose}, a platform for visualization and real-time monitoring of MapReduce data workflows. They offer three different views to show associated jobs, job dependencies and progress. Ambrose has been released as open source. Ambrose, however, is not suitable for our needs as the abstraction level of jobs is too high and does not capture single operators.

Google's Dapper\cite{sigelman2010dapper} is a distributed systems tracing infrastructure and offers fine grained tracing of calls in Google's distributed systems. They also proposed an interface for visualizing traces. Similarly, X-trace\cite{fonseca2007x} was developed as a framework to trace which events cause what other events in a distributed environment. recently, there has been work on visualizing event traces collected in X-trace\footnote{\url{https://github.com/brownsys/X-Trace/tree/master/src/webui/html/interactive}}. Due to the importance of these kinds of debug facilities, Twitter closely modeled Zipkin\cite{zipkin} after Dapper and X-trace and released it as open source.

Dapper and X-trace focus on how data flows through a distributed system. In \system the visualization focuses on the operators and how the data flows through them. This orthogonal view is better suited for debugging performance bottlenecks in DDBMS and also scales to a larger number of events. Furthermore, in contrast to  we offer different abstraction levels, which enables users to find problems faster and handle larger amounts of profiling data. Also, \system is specifically designed to help developers understand query execution in distributed database system like Myria as opposed to general traces in distributed systems.

Tools to visualize query plans for example in SQL Server, as used to improve performance for the SDSS Sky survey\cite{szalay2002sdss}, focus on optimizing queries and not query execution and have no visualization of data flow, which is necessary to optimize physical query execution.

\section{Approach}

% Adriana

The first step in designing \system was to identify the possible causes that
could affect the execution performance of the query. After we identified
several such causes, that we will enumerate below, we leveraged visualization
techniques that we found fit to allow developers/programmers to get insight
into how the query was executed on the available resources and quickly
identify the source(s) of the performance penalty.

We found that the performance of the query might be affected by the following
factors, which influenced the design of our visualization:
\begin{itemize}
   \item Wrong/unoptimized physical query plan. 
   \item Stragglers,   
\end{itemize} 

% Why did we choose the visualization

\begin{itemize}
    \item determine possible causes (data skew, stragglers, etc.) for why a query might be slow and design visualizations to
help
    \item we designed three views - physical query plan, fragment view, communication view.
\end{itemize}


\system's frontend is embedded into Myria's web frontend and provides comprehensive performance views to the user. There are 4 views in total that allow the user to visualize query performance details under different angles:
\begin{itemize}
    \item Physical Query Plan view: this view allows the user to connect the query she wrote with the physical query plan that the query optimizer generates.
    \item etc. 
\end{itemize}

\subsection{Back-end}

% Dom

\subsection{Front-end}
The Myria web front-end server is written in Python and runs on Google App Engine. \system's user-interface (UI) is embedded into Myria's web front-end. We build \system's UI using D3 to delive an interactive visualization experience to the user. D3 is a JavaScript framework for data visualziation on the web. We used visualization techniques such as empth{focus+context} that are possible to implement conveniently using D3.
The \system's front-end gets data from the Myria server in JSON and CSV file formats. JSON objects are used to describe the physical queryplan, while CSV files contain information describing the execution of the query at the fragment and operator level, the total amount of data exchange between workers at different stages of the query execution.
The \system's web UI is divided into two main sections.
% Thierry



\subsubsection{Overview over all fragments}

% Adriana

% Write about small multiples view

\subsubsection{Physical query plan view}

% Thierry

\subsubsection{Fragment execution view}

% Adriana

\subsubsection{Worker communication view}

% Umar

\section{Evaluation}

% How it helps
% Examples

% Thierry + Dom

\textbf{Discussion:}

\section{Conclusion}

% Adriana

\section{Future Work}

% Dom

\bibliographystyle{abbrv}
\bibliography{paper}

\end{document}
